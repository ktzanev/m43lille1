\documentclass[a4paper,10.5pt,french]{article}
\usepackage[T1]{fontenc}
\usepackage[ansinew]{inputenc}
\usepackage{babel}
\usepackage[T1]{fontenc}
\usepackage[ansinew]{inputenc}
\usepackage{babel,amsmath,amssymb}
\newcommand{\K}{\rm I\!K}
\newcommand{\N}{\rm I\!N}
\newcommand{\Z}{\mathbf {Z}}
\newcommand{\Q}{\mathbf {Q}}
\newcommand{\R}{\rm I\!R}
\newcommand{\C}{\mathbf {C}}
\newcommand{\D}{\mathcal {D}}
\newcommand{\NN}{\mathcal {N}}
\newcommand{\1}{\ensuremath{\mbox{\rm 1\kern-0.23em I}}}

\addtolength{\oddsidemargin}{-2cm}
\addtolength{\evensidemargin}{-2cm}
\addtolength{\textwidth}{4cm}
\addtolength{\topmargin}{-2cm}
\addtolength{\textheight}{4cm}
\begin{document}

\begin{flushleft}

{\bf Universit\'e des Sciences et Technologies de Lille}, 
{\bf UFR de Math�matiques Pures et Appliqu�es},
{Ann\'ee 2014-2015}
 \end{flushleft} 
  {\bf Licence MIMP S4}, Module: Probabilit�s discr�tes\\
 
 \hrule
 \vspace{1cm}
 
\hskip 3cm {\large Fiche 2: Probabilit�s conditionnelles et ind�pendance }\\



{\bf Exercice 1:} 

 Une urne contient $10$ jetons jaunes, $5$ blancs et $1$ rouge. 
J'ai tir� un jeton de cette urne et je vous annonce qu'il n'est pas rouge. 
Quelle est la probabilit� qu'il soit jaune?\\



{\bf Exercice 2:}

Un joueur de tennis a une probabilit\'e de $40\%$ de passer sa premi\`ere balle
de service. S'il \'echoue, sa probabilit\'e de passer sa deuxi\`eme balle est
$70\%$. Lorsque sa premi\`ere balle de service passe, sa probabilit\'e de
gagner
le point est $80\%$, tandis que sa probabilit\'e de gagner le point lorsqu'il
passe sa deuxi\`eme balle de service n'est plus que $50\%$. Calculer

1)
La probabilit\'e qu'il passe sa deuxi\`eme
balle et celle qu'il fasse une double faute.

2)
La probabilit\'e qu'il perde le point sur son service.

3)
Sachant qu'il a perdu le point, quelle est la probabilit\'e que ce soit
sur une double faute ?\\



{\bf Exercice 3:} 

On cherche une girafe qui, avec une probabilit\'e $p/7$, se trouve dans l'un des quelconques des $7$ \'etages d'un immeuble, et avec probabilit\'e $1-p$ hors de l'immeuble. On a explor� en vain les $6$ premiers �tages.

1)
Quelle est la probabilit\'e qu'elle habite au  septi\`eme \'etage?
On note $f(p)$ cette probabilit\'e.

2)
Repr\'esenter  la fonction $p\rightarrow f(p)$\\



{\bf Exercice 4:} 

Un test permet de d�pister si une pi�ce est d�fectueuse. Il n'est
cependant pas fiable absolument. Ce test donne pour pi�ce d�fectueuse
une pi�ce d�fectueuse dans 95 $\%$ des cas et une pi�ce non
d�fectueuse pour une pi�ce saine dans 90 $\%$ des cas. Un lot de
pi�ces contient 8 $\%$ de pi�ces d�fectueuses. On prend une des pi�ces
du lot au hasard : la pi�ce choisie est test�e et d�clar�e d�fectueuse. Quelle est la probabilit� pour qu'elle le soit vraiment ?\\


{\bf Exercice 5:}


Un avion a disparu et la r\'egion  o\`u il s'est \'ecras\'e est divis\'ee pour sa 
recherche en trois zones de m\^eme probabilit\'e. Pour $i=1,2,3$, notons
$1-\alpha_i$ la probabilit\'e que l'avion soit retrouv\'e par une recherche
dans la zone $i$ s'il est effectivement dans cette zone. Les constantes
$\alpha_i$ repr\'esentent les probabilit\'es de manquer l'avion et sont
g\'en\'eralement attribuables \`a l'environnement de la zone (relief,
v\'eg\'etation,\dots). On notera $A_i$ l'\'ev\'enement \emph{l'avion est dans la
zone $i$}, et $R_i$ l'\'ev\'enement \emph{l'avion est retrouv\'e dans la zone
$i$} ($i=1,2,3$). 

1)
 Pour $i=1,2,3$, d\'eterminer les probabilit\'es que l'avion soit dans la 
zone $i$ sachant que la recherche dans la zone 1 a \'et\'e infructueuse.

2)
 Etudier bri\`evement les variations de ces trois probabilit\'es 
conditionnelles consid\'er\'ees comme fonctions de $\alpha_1$ et commenter les
r\'esultats obtenus.\\


{\bf Exercice 6:}


On lance deux d\'es et on consid\`ere les \'ev\'enements :

$A$ : ``le r\'esultat du premier d\'e est impair"

$B$ : ``le r\'esultat du second d\'e est pair"

$C$ : ``les r\'esultats des deux d\'es sont de m\^eme parit\'e"\\
Etudier l'ind\'ependance deux \`a deux des \'ev\'enements $A$, $B$ et $C$, 
puis l'ind\'ependance mutuelle (ind\'ependance de la famille) $A,B,C$.\\





{\bf Exercice 7:}


On s'int\'eresse \`a la r\'epartition des sexes des enfants d'une famille de
$n$ enfants. On prend comme mod\'elisation
$$ \Omega_n=\{{\tt f},{\tt g}\}^n=\{\,(x_1,\ldots,x_n),\; 
x_i\in\{{\tt f},{\tt g}\}, i=1,\dots,n\},
$$
muni de l'\'equiprobabilit\'e. On consid\`ere les \'ev\`enements:
$$
A=\{\mbox{la famille a des enfants des deux sexes}\}\quad
B=\{\mbox{la famille a au plus une fille}\}.
$$

1)
 Montrer que pour $n\geq 2$, $P(A)=(2^n-2)/2^n$ et $P(B)=(n+1)/2^n$.
 
2)
 En d\'eduire que $A$ et $B$ ne sont ind\'ependants que si $n=3$.\\

{\bf Exercice  8:}

On effectue des lancers r\'ep\'et\'es d'une paire de d\'es discernables et on observe pour
chaque lancer la \emph{somme} des points indiqu\'es par les deux d\'es. On se
propose de calculer de deux fa\c{c}ons la probabilit\'e de l'\'ev\'enement $E$ d\'efini
ainsi: \emph{dans la suite des r\'esultats observ\'es, la premi\`ere obtention d'un
$9$ a lieu avant la premi\`ere obtention d'un $7$.}

1)
 Quelle
est la probabilit\'e de n'obtenir ni $7$ ni $9$ au cours d'un lancer? 

2)
\emph{Premi\`ere m\'ethode:}  
On note $F_i=\{$\emph{obtention d'un $9$ au $i$-\`eme lancer}$\}$
 et
pour $n>1$,  
$E_n=\{$ \emph{ni $7$ ni $9$ ne sont obtenus au cours
des $n-1$ premiers lancers et le $n$-i\`eme lancer donne $9$}$\}$. Dans le
cas particulier $n=1$, on pose $E_1=F_1$. 
\begin{itemize}
\item[a)]
Exprimer $E$ \`a l'aide d'op\'erations
ensemblistes sur les $E_n$ ($n\geq 1$). Exprimer de m\^eme chaque $E_n$ \`a
l'aide  des $F_i$ et des  $H_i=\{$ \emph{ni $7$ ni $9$ au $i$-\`eme
lancer}$\}$. 
\item[b)]
 Calculer $P(E_n)$ en utilisant l'ind\'ependance des lancers.
\item[c)]
 Calculer $P(E)$.
\end{itemize}

3)
\emph{Deuxi\`eme m\'ethode:}  On note 
$G_1=\{$ \emph{obtention d'un $7$ au premier lancer}$\}$. 
\begin{itemize}
\item[a)]
Donner une
expression de $P(E)$ en utilisant le conditionnement par la partition
$\{F_1,G_1,H_1\}$. 
\item[b)]
 Donner sans calcul les valeurs de $P(E\mid F_1)$, $P(E\mid G_1)$ et
expliquer pourquoi $P(E\mid H_1)=P(E)$.
\item[c)] 
En d\'eduire la valeur de $P(E)$.
\end{itemize}


\end{document}

