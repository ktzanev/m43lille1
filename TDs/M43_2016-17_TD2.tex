\documentclass[a4paper,12pt,reqno]{amsart}
\usepackage{macros_M43}

\begin{document}

% ===================================================================
\hautdepage{Fiche 2: Probabilités conditionnelles et indépendance}
% ===================================================================


%-----------------------------------
\begin{exo}

  Une urne contient $10$ jetons jaunes, $5$ blancs et $1$ rouge. J'ai tiré un jeton de cette urne et je vous annonce qu'il n'est pas rouge. Quelle est la probabilité qu'il soit jaune?

\end{exo}

%-----------------------------------
\begin{exo}

  Un joueur de tennis a une probabilité de $40\%$ de passer sa première balle de service. S'il échoue, sa probabilité de passer sa deuxième balle est $70\%$. Lorsque sa première balle de service passe, sa probabilité de gagner le point est $80\%$, tandis que sa probabilité de gagner le point lorsqu'il passe sa deuxième balle de service n'est plus que $50\%$. Calculer

  \begin{enumerate}
    \item La probabilité qu'il passe sa deuxième balle et celle qu'il fasse une double faute.
    \item La probabilité qu'il perde le point sur son service.
    \item Sachant qu'il a perdu le point, quelle est la probabilité que ce soit sur une double faute ?
  \end{enumerate}

\end{exo}

%-----------------------------------
\begin{exo}

  On cherche une girafe qui, avec une probabilité $p/7$, se trouve dans l'un des quelconques des $7$ étages d'un immeuble, et avec probabilité $1-p$ hors de l'immeuble. On a exploré en vain les $6$ premiers étages.

  \begin{enumerate}
    \item Quelle est la probabilité qu'elle habite au  septième étage? On note $f(p)$ cette probabilité.
    \item Représenter  la fonction $p\mapsto f(p)$
  \end{enumerate}

\end{exo}

%-----------------------------------
\begin{exo}

  On étudie une maladie qui touche 1$\%$ d'une population. On dispose d'un test  de dépistage de la maladie qui a les caractéristiques suivantes:
  \begin{itemize}
    \item Si la personne est malade, le test est positif avec une probabilité de $99 \%$.
    \item Si la personne n'est pas malade, le test est positif avec une probabilité de $9 \%$.
  \end{itemize}
  Dans la suite, on notera $M$ l'événement «la personne est malade» et $T$ l'événement «le test est positif».

  \begin{enumerate}
    \item Traduire les données de l'énoncé en donnant, sans aucun calcul,  la valeur de chacune des quantités suivantes : $P_M(T),$ $P_{\overline M}(T)$ et $P(M)$.
    \item Calculer $P(T)$.
    \item Une personne a effectué le test, qui est positif. Quelle est la probabilité qu'elle soit malade ?
    \item En déduire le taux de faux positifs pour ce test médical.
    \item Quel est le taux de faux négatifs ?
  \end{enumerate}

\end{exo}

%-----------------------------------
\begin{exo}

  Un avion a disparu et la région  où il s'est écrasé est divisée pour sa
  recherche en trois zones de même probabilité. Pour $i=1,2,3$, notons
  $1-\alpha_i$ la probabilité que l'avion soit retrouvé par une recherche
  dans la zone $i$ s'il est effectivement dans cette zone. Les constantes
  $\alpha_i$ représentent les probabilités de manquer l'avion et sont
  généralement attribuables à l'environnement de la zone (relief,
  végétation,\dots). On notera $A_i$ l'événement \emph{l'avion est dans la
  zone $i$}, et $R_i$ l'événement \emph{l'avion est retrouvé dans la zone
  $i$} ($i=1,2,3$).

  \begin{enumerate}
    \item  Pour $i=1,2,3$, déterminer les probabilités que l'avion soit dans la zone $i$ sachant que la recherche dans la zone 1 a été infructueuse.
    \item  Étudier brièvement les variations de ces trois probabilités conditionnelles considérées comme fonctions de $\alpha_1$ et commenter les résultats obtenus.
  \end{enumerate}

\end{exo}

%-----------------------------------
\begin{exo}

  On lance deux dés et on considère les événements :
  \begin{eqnarray*}
    A & = & \{\text{le résultat du premier dé est impair}\},\\
    B & = & \{\text{le résultat du second dé est pair}\},\\
    C & = & \{\text{les résultats des deux dés sont de même parité}\}.
  \end{eqnarray*}
  Etudier l'indépendance deux à deux des événements $A$, $B$ et $C$,
  puis l'indépendance mutuelle (indépendance de la famille) $A,B,C$.

\end{exo}

%-----------------------------------
\begin{exo}

  On s'intéresse à la répartition des sexes des enfants d'une famille de
  $n$ enfants. On prend comme modélisation
    $$
      \Omega_n=\{{\tt f},{\tt g}\}^n=\{\,(x_1,\ldots,x_n),\; x_i\in\{{\tt f},{\tt g}\}, i=1,\dots,n\},
    $$
  muni de l'équiprobabilité. On considère les événements:
    $$
      A=\{\text{la famille a des enfants des deux sexes}\}\quad
      B=\{\text{la famille a au plus une fille}\}.
    $$

  \begin{enumerate}
    \item  Montrer que pour $n\geq 2$, $P(A)=(2^n-2)/2^n$ et $P(B)=(n+1)/2^n$.
    \item  En déduire que $A$ et $B$ ne sont indépendants que si $n=3$.
  \end{enumerate}

 \end{exo}

%-----------------------------------
\begin{exo}

  On effectue des lancers répétés d'une paire de dés discernables et on observe pour
  chaque lancer la \emph{somme} des points indiqués par les deux dés. On se
  propose de calculer de deux façons la probabilité de l'événement $E$ défini
  ainsi: \emph{dans la suite des résultats observés, la première obtention d'un
  $9$ a lieu avant la première obtention d'un $7$.}

  \begin{enumerate}
    \item  Quelle est la probabilité de n'obtenir ni $7$ ni $9$ au cours d'un lancer?
    \item \emph{Première méthode:} On note $F_i=\{$\emph{obtention d'un $9$ au $i$-ième lancer}$\}$ et pour $n>1$, $E_n=\{$ \emph{ni $7$ ni $9$ ne sont obtenus au cours des $n-1$ premiers lancers et le $n$-ième lancer donne $9$}$\}$. Dans le cas particulier $n=1$, on pose $E_1=F_1$.
      \begin{enumerate}
        \item Exprimer $E$ à l'aide d'opérations ensemblistes sur les $E_n$ ($n\geq 1$). Exprimer de même chaque $E_n$ à l'aide  des $F_i$ et des  $H_i=\{$ \emph{ni $7$ ni $9$ au $i$-ième lancer}$\}$.
        \item Calculer $P(E_n)$ en utilisant l'indépendance des lancers.
        \item Calculer $P(E)$.
      \end{enumerate}
    \item \emph{Deuxième méthode:}  On note $G_1=\{$ \emph{obtention d'un $7$ au premier lancer}$\}$.
      \begin{enumerate}
        \item Donner une expression de $P(E)$ en utilisant le conditionnement par la partition $\{F_1,G_1,H_1\}$.
        \item Donner sans calcul les valeurs de $P_{F_1}(E)$, $P_{G_1}(E)$ et expliquer pourquoi $P_{H_1}(E)=P(E)$.
        \item En déduire la valeur de $P(E)$.
      \end{enumerate}
  \end{enumerate}

\end{exo}

\end{document}

