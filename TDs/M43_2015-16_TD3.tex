\documentclass[a4paper,12pt,reqno]{amsart}
\usepackage{macros_M43}

\begin{document}

% ===================================================================
\hautdepage{Fiche 3: Variables aléatoires discrètes}
% ===================================================================


%-----------------------------------
\begin{exo}

  On jette un dé bleu et un dé vert non pipés et on considère la v.a. $X$ égale à
  la somme des points obtenus. Quelle est la loi de $X$? Même question pour la v.a. $Y$ égale au minimum des deux points obtenus.

\end{exo}

%-----------------------------------
\begin{exo}

  Une urne contient $N$ jetons numérotés de 1 à $N$. On en tire $n\leq N$ au hasard et sans remise. Soit $k\in \{1,\ldots,N\}$.

  \begin{enumerate}
    \item Décrire  un espace de probabilité $(\Omega,\mathcal{P}(\Omega),P)$  associé à l'expérience aléatoire.
    \item Quelle est la probabilité que les jetons tirés aient tous des numéros inférieurs ou égaux à $k$?
    \item On désigne par $X$ la variable égale au plus grand numéro des jetons tirés. Déterminer la loi de $X$.
    \item Mêmes questions avec un tirage avec remise.
  \end{enumerate}

\end{exo}

%-----------------------------------
\begin{exo}

  On lance ensemble 5 dés. On met de coté ceux qui ont donné l'as, puis on relance les autres, et ainsi de suite jusqu'à ce que tous les dés aient amené l'as. Le jeu s'arrête au bout de $ X $ lancers.

  \begin{enumerate}
    \item Les dés sont numérotés, et pour $ i $ allant de $ 1 $ à $ 5 $ on note $ X_i $ le nombre de fois où le dé $ i $ a été lancé. Quelle est la loi de $ X_i $? Donner sa fonction de répartition $ F_i $.
    \item Exprimer $ X $ en fonction des $ X_i $ et déterminer sa loi (il est conseillé d'exprimer la fonction de répartition $ F $ de $ X $ en fonction des $ F_i $).
  \end{enumerate}

\end{exo}

%-----------------------------------
\begin{exo}

  Quatre chasseurs tirent indépendamment les uns des autres  chacun un coup de fusil  sur un bison en fuite. Chaque chasseur a une chance sur 4 de toucher le bison. On suppose qu'il faut au moins 3 balles pour tuer un bison.

  \begin{enumerate}
    \item Quelle est la loi de la v.a. $X$ égale au nombre de balles touchant le bison?
    \item Quelle est la probabilité que l'animal soit touché?
    \item Quelle est la probabilité qu'il soit tué?
    \item Sachant qu'il a été touché, quelle est la probabilité qu'il soit simplement blessé?
  \end{enumerate}

\end{exo}

%-----------------------------------
\begin{exo}

  Un moteur d'avion tombe en panne avec la probabilité $1-p$,~~ $p \in ]0,1[$, indépendamment des autres moteurs. Pour réussir le vol l'avion a besoin que la majorité de ses moteurs fonctionne correctement (sans panne).

  \begin{enumerate}
    \item Calculer la probabilité de réussir le vol pour un avion à $3$ moteurs.
    \item Calculer la probabilité de réussir le vol pour un avion à $5$ moteurs.
    \item Déterminer pour quelle valeur de $p$ l'avion à $5$ moteurs est préférable à l'avion à $3$ moteurs.
  \end{enumerate}

\end{exo}

%-----------------------------------
\begin{exo}

  Soit $X$ une v.a. de Poisson de paramètre $\lambda>0$. On définit la v.a. $Y$ de la manière suivante:
  \begin{enumerate}
    \item Si $X$ prend une valeur nulle ou impaire alors $Y$ prend la valeur $0$.
    \item Si $X$ prend une valeur paire alors $Y$ prend la valeur $X/2$.
  \end{enumerate}
  Trouver la loi de $Y$.

\end{exo}

%-----------------------------------
\begin{exo}

  \begin{enumerate}
    \item Si $X$ suit la loi de Poisson de paramètre $\lambda>0$, montrer que:
      $$
        \forall k>\lambda-1,\qquad P(X\geq k)<P(X=k)\frac{k+1}{k+1-\lambda}.
      $$
      \begin{indication}
      On part de:
        $$
          \sum_{j=k}^{+\infty}\frac{\lambda^j}{j!}=
            \frac{\lambda^k}{k!}\Bigl(1+\frac{\lambda}{k+1}+
              \frac{\lambda^2}{(k+1)(k+2)}+\frac{\lambda^3}{(k+1)(k+2)(k+3)}
                + \cdots\cdots\Bigr)
        $$
      et on majore la parenthèse par la somme d'une série géométrique\dots
      \end{indication}
    \item En déduire que:
      $$
        \forall k\geq 2\lambda-1,\qquad P(X>k)<P(X=k).
      $$
  \end{enumerate}

\end{exo}
%-----------------------------------
\begin{exo}

  \begin{enumerate}
    \item Quelle est la probabilité d'obtenir un double lorsqu'on lance une fois une paire de dés?
    \item On lance de façon répétée une paire de dés jusqu'à la première obtention d'un double. Soit $X$ la variable aléatoire égale au nombre de lancers ainsi réalisés. Donner, en justifiant votre réponse, les valeurs de $P(X=1)$, $P(X=2)$ et plus généralement $P(X=k)$ où $k\geq 1$ est un entier quelconque. Comment s'appelle la loi de $X$?
    \item Soit $n\geq 1$ un nombre entier. Calculer \emph{directement} la probabilité de n'obtenir aucun double lors des $n$ premiers lancers. En déduire $P(X>n)$.
    \item Pour $k>n$, vérifier que
      $$
        P(X=k \mid X>n)=P(X=k-n).
      $$
    Donner une interprétation.
  \end{enumerate}

\end{exo}
%-----------------------------------
\begin{exo}

  \emph{
    Dans cet exercice, on précisera pour les variables aléatoire $X$, $V$ et $Y$:
    \begin{itemize}
      \item les valeurs que peut prendre la variable aléatoire,
      \item leur loi ainsi que le nom de leur loi s'il existe.
    \end{itemize}
  }

  \begin{enumerate}
    \item Monsieur Zzzz possède 50 cravates, dont une seule est à rayures. Tous les matins, il prend une cravate au hasard dans l'armoire, et tous les soirs il remet la cravate du jour à sa place.\\
    On observe Monsieur Zzzz pendant 20 jours et on appelle $ X $ le nombre de fois où il porte une cravate à rayures. \\
    Application numérique: Calculer $ P(X=1) $.
    \item Monsieur Zzzz part en voyage. Il met dans sa valise 20 cravates prises au hasard dans l'armoire. \\
    On appelle $ V $ le nombre de cravates à rayures contenues dans la valise? \\
    Application numérique: Calculer $ P(V=1) $.
    \item Monsieur Zzzz possède aussi 10 chemises dont 3 sont bleues. Il prend 5 chemises au hasard et les mets dans sa valise. \\
    On appelle $ Y $ le nombre de chemises bleues contenues dans la valise? \\
    Application numérique: Calculer $ P(Y=1) $.\\
  \end{enumerate}
\end{exo}
%-----------------------------------
\begin{exo}

  Un écran d'ordinateur est formé de petits points lumineux appelés pixels. Il comporte 768 lignes de 1024 pixels, soit 786432 pixels en tout.
  \begin{enumerate}
      \item On utilise un procédé de fabrication qui assure que les pixels sont indépendants et que chacun n'a qu'une probabilité $ 9.10^{-7} $ d'être inutilisable. Quelle est la loi du nombre $ X $ de pixels grillés sur l'écran?
      \item L'écran est invendable si trois pixels au moins sont grillés. Calculer (en justifiant!) une valeur approchée de la probabilité pour un écran d'être invendable.
  \end{enumerate}
\end{exo}
%-----------------------------------
\begin{exo}

  Une entreprise fabrique des chaises à roulettes, équipées chacune de 5 roulettes.

  \begin{enumerate}
    \item Par suite d'une erreur de livraison, l'entreprise a reçu 10\,000 roulettes en bon état et 1\,000 roulettes présentant un défaut de fabrication. Parmi les 2\,200 chaises ainsi fabriquées, on en teste une au hasard. Calculer avec précision la probabilité que ses cinq roulettes soient en bon état.
    \item Quelle est la loi du nombre $ X $ de roulettes défectueuses dont est munie la chaise testée? Par quelle loi peut-on l'approcher? En utilisant cette approximation, évaluer la probabilité que la chaise testée ait exactement une roulette défectueuse, et la probabilité qu'elle en ait exactement trois.
    \item En utilisant la même approximation, calculer la probabilité que la chaise soit en bon état et comparer avec le résultat de la première question.
  \end{enumerate}

\end{exo}
\end{document}

