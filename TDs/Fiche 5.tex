\documentclass[a4paper,10.5pt,french]{article}
\usepackage[T1]{fontenc}
\usepackage[ansinew]{inputenc}
\usepackage{babel}
\usepackage[T1]{fontenc}
\usepackage[ansinew]{inputenc}
\usepackage{babel,amsmath,amssymb}
\newcommand{\K}{\rm I\!K}
\newcommand{\N}{\rm I\!N}
\newcommand{\Z}{\mathbf {Z}}
\newcommand{\Q}{\mathbf {Q}}
\newcommand{\R}{\rm I\!R}
\newcommand{\C}{\mathbf {C}}
\newcommand{\D}{\mathcal {D}}
\newcommand{\NN}{\mathcal {N}}
\newcommand{\1}{\ensuremath{\mbox{\rm 1\kern-0.23em I}}}

\addtolength{\oddsidemargin}{-2cm}
\addtolength{\evensidemargin}{-2cm}
\addtolength{\textwidth}{4cm}
\addtolength{\topmargin}{-2cm}
\addtolength{\textheight}{4cm}
\begin{document}

\begin{flushleft}

{\bf Universit\'e des Sciences et Technologies de Lille}, 
{\bf UFR de Math�matiques Pures et Appliqu�es},
{Ann\'ee 2014-2015}
 \end{flushleft} 
  {\bf Licence MIMP S4}, Module: Probabilit�s discr�tes\\
 
 \hrule
 \vspace{1cm}
 
\hskip 3cm {\large Fiche 5: Moments de variables al�atoires r�elles }\\



{\bf Exercice 1:} 

On lance un d\'e �quilibr�. On d\'esigne par $X$ le nombre obtenu avec ce d\'e.

1) D\'eterminer l'esp\'erance et la variance de $X$.

2) Calculer l'esp\'erance et la variance de la somme obtenue en lan\c{c}ant trois d\'es. Comparer aux valeurs de l'esp�rance et de la variance de trois fois la valeur d'un d�.\\

{\bf Exercice 2:}

Alice et Bob jouent aux d\'es.
La partie est d\'ej\`a bien engag\'ee, et \`a ce stade, 
c'est \`a Alice de lancer les deux d\'es. 
Si au moins un des d\'es donne cinq, elle gagne, et
Bob doit lui donner 11 Euros. 
Si elle n'obtient pas de cinq, elle perd, et elle doit alors 11 Euros \`a son adversaire. 
Bob propose~: \\
``tu ne lances pas les d\'es, tu me donnes directement 4 Euros et nous sommes quittes.'' \\
On veut d\'eterminer si Alice a int\'er\^et \`a accepter ce march\'e.

1) On note $ X $ la somme (positive ou n\'egative) que va gagner Bob.
Quelle est la loi de $ X $~?

2) Calculer son esp\'erance $ E(X) $.
Comparer ce gain moyen avec la somme de 4 euros que Bob r\'eclame 
et conclure.\\

{\bf Exercice 3:} 

Soit $X$ une variable al\'eatoire \`a valeurs dans $\N^*$ admettant une 
esp�rance. 

1) Montrer que l'on a $\displaystyle E(X)=\sum_{k=1}^\infty P(X \geq k) \, .$ 
Donner et d�montrer une formule analogue pour $E(X^2)$ (si $X^2$ admet une esp�rance).

2) Soit $X$ est une variable al\'eatoire de loi g\'eom\'etrique de param\`etre $p$ ($p  \in ]0,1[$), utiliser la question pr�c�dente pour retrouver l'esp\'erance de $X$.\\

{\bf Exercice 4:} 

Dans une urne contenant au d\'epart une boule verte et une rouge on 
effectue une suite de tirages d'une boule selon la proc\'edure suivante.
Chaque fois que l'on tire une boule verte, on la remet dans l'urne
{\em en y rajoutant une boule rouge}. Si l'on tire une boule rouge, on
arr\^ete les tirages. On d\'esigne par $X$ le nombre de tirages effectu\'es
par cette proc\'edure. On notera $V_i$ (resp. $R_i$) l'\'ev\'enement {\em
obtention d'une boule verte au $i$-\`eme tirage} (resp. {\em rouge}).  

1) Pour $k\in\N^{\ast}$, donner une expression de l'\'ev\'enement
$\{X=k\}$ \`a l'aide des \'ev\'enements $V_i$ ($1\leq i\leq k-1$) et $R_k$.

2) Que vaut $P(V_n\mid V_1\cap\ldots\cap V_{n-1})$ pour $n\geq 2$? 

3) D\'eterminer la loi de $X$. 

4) Calculer $E(X)$.

5) Calculer l'esp\'erance de la variable al\'eatoire $\frac{1}{X}$.

6) On recommence l'exp\'erience en changeant la proc\'edure: \`a chaque 
tirage d'une boule verte on la remet dans l'urne {\em en y rajoutant
une boule verte}. Comme pr\'ec\'edemment, on interrompt les tirages \`a la
premi\`ere apparition d'une boule rouge. Soit $Y$ le nombre de tirages
effectu\'es suivant cette nouvelle proc\'edure. Trouver la loi de $Y$. Que
peut-on dire de l'esp\'erance de $Y$? Interpr\'eter.\\

{\bf Exercice 5:}


Soit $X$ une variable al�atoire discr�te suivant une loi de Poisson de
param�tre $\lambda>0$. Justifier que l'esp�rance de la variable al�atoire
discr�te $Y=e^X$ existe  et la calculer. On dit que $X$ admet un moment exponentiel.\\

\newpage

{\bf Exercice 6:}

Soit $X$ une variable al\'eatoire, prenant toutes les valeurs
enti\`eres comprises entre $-5$ et $5$, dont la loi est d\'efinie par~:
$$
P(X=i-5)=C_{10}^i\Bigl(\frac{1}{2}\Bigr)^{10}\qquad (0\leq i\leq 10).
$$

1) Calculer $E(X)$, $E(X^2)$, $Var(X)$.

2) Donner la loi, l'esp\'erance, la variance de la variable al\'eatoire
$Y$, d\'efinie par : $Y=1$ si $|X| > 2$, $Y=0$ sinon.\\

{\bf Exercice 7:}

Soit $X$ une v.a. telle que $E(X)=1$ et $\text{Var}(X)=5$. \\
Trouver

1) $E(2+X)^2$;

2) $\text{Var}(4-3X)$.\\

{\bf Exercice 8:}

Soit $X$ et $Y$ deux variables al\'{e}atoires ind\'{e}pendantes de loi de 
Bernoulli de param\`{e}tre 1/2. 
On d�finit les variables al\'{e}atoires $S=X+Y$ et $D=|X-Y|$. 

1) Donner les lois de $S$ et $D$.

2) Calculer $Cov(S,D)$. Les variables al\'{e}atoires $S$ et $D$ sont-elles ind\'{e}pendantes ?\\


{\bf Exercice 9:}

On consid\`ere une suite de $n$ \'epreuves r\'ep\'et\'ees ind\'ependantes, chaque
\'epreuve ayant $k$ r\'esultats possibles $r_1,\ldots,r_k$. 
On note $p_i$ la probabilit\'e de r\'ealisation du r\'esultat $r_i$ lors 
d'une \'epreuve donn\'ee. 
Par exemple si on lance $n$ fois un d\'e, $k=6$ et $p_i=1/6$ pour 
$1\leq i\leq 6$. 
Si on lance $n$ fois une paire de d\'es et que l'on s'int\'eresse \`a la somme 
des points, $k=11$ et les $p_i$ sont donn\'es par le tableau des $P(S=i+1)$~:
$p_1=1/36$, $p_2=2/36$,\dots \\
Pour $1\leq i\leq k$, notons $X_i$ le nombre de r\'ealisations du
r\'esultat $r_i$ au cours des $n$ \'epreuves.

1) Expliquer sans calcul pourquoi $Var(X_1+\cdots +X_k)=0$.

2) Quelle est la loi de $X_i$? Que vaut sa variance?

3) Pour $i\neq j$, donner la loi et la variance de $X_i+X_j$.

4) En d\'eduire $Cov(X_i,X_j)$.

5) Contr\^oler ce r\'esultat en d\'eveloppant $Var(X_1+\cdots +X_k)$ et
en utilisant la premi\`ere question.


\end{document}

