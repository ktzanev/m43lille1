\documentclass[a4paper,10.5pt,french]{article}
\usepackage[T1]{fontenc}
\usepackage[ansinew]{inputenc}
\usepackage[francais]{babel}
\usepackage{babel,amsmath,amssymb}
\newcommand{\K}{\rm I\!K}
\newcommand{\N}{\rm I\!N}
\newcommand{\Z}{\mathbf {Z}}
\newcommand{\Q}{\mathbf {Q}}
\newcommand{\R}{\rm I\!R}
\newcommand{\C}{\mathbf {C}}
\newcommand{\D}{\mathcal {D}}
\newcommand{\NN}{\mathcal {N}}
\newcommand{\1}{\ensuremath{\mbox{\rm 1\kern-0.23em I}}}

\addtolength{\oddsidemargin}{-2cm}
\addtolength{\evensidemargin}{-2cm}
\addtolength{\textwidth}{4cm}
\addtolength{\topmargin}{-2cm}
\addtolength{\textheight}{4cm}
\begin{document}

\begin{flushleft}

{\bf Universit\'e des Sciences et Technologies de Lille} 
{\bf UFR de Math�matiques Pures et Appliqu�es},
{Ann\'ee 2014-2015}
 \end{flushleft} 
  {\bf Licence MIMP S4}, Module: Probabilit�s discr�tes\\
 
 \hrule
 \vspace{1cm}
 
\hskip 4cm {\large Fiche 1: Espaces probabilis�s }\\



{\bf Exercice 1:} 


Trois enfants lancent chacun un ballon en direction d'un panier de basket.
Il est convenu que celui qui marquera gagnera un paquet de bonbons, et 
qu'en cas d'ex-\ae quo les vainqueurs se partageront le paquet.
Si personne ne r\'eussit son panier, chacun mangera le tiers des bonbons. 
En utilisant les \'ev\'enements:
\begin{eqnarray*}
A & = & \{\textrm{Arthur marque un panier}\},\\
B & = & \{\textrm{B\'eatrice marque un panier}\},\\
C & = & \{\textrm{C\'ecile marque un panier}\}.
\end{eqnarray*}
\'ecrire de fa\c con ensembliste les \'ev\'enements suivants:
\begin{eqnarray*}
D & = & \{\textrm{tous les trois r\'eussissent \`a marquer}\},\\
E & = & \{\textrm{aucun ne r\'eussit \`a marquer}\},\\
F & = & \{\textrm{B\'eatrice mange tous les bonbons}\},\\
G & = & \{\textrm{les trois enfants mangent des bonbons}\},\\
H & = & \{\textrm{C\'ecile mange au moins un bonbon}\},\\
I & = & \{\textrm{Arthur ne re\c coit aucun bonbon}\}.
\end{eqnarray*}
Parmi tous ces \'ev\'enements, lesquels sont des \'ev\'enements \'el\'ementaires~?\\


{\bf Exercice 2:}

 On  lance un d\'e jusqu'\`a la premi\`ere
apparition du six. Notons:
$$ S_i=\{\textrm{Le $i$-\`eme lancer donne un six}\},\qquad i\in\N.$$
Ecrire \`a l'aide des \'ev\'enements $S_i$ et $S_i^c$ l'\'ev\'enement
$$ A=\{\textrm{La premi\`ere apparition du six a lieu \emph{apr\`es} le 5-\`eme
lancer}\}.$$
Est-ce le m\^eme \'ev\'enement que 
$$
B=\{\textrm{Six n'appara\^{\i}t pas au cours des 5 premiers lancers}\}?$$




{\bf Exercice 3:} 

Soit $n\in \N^*$. On tire \`{a} pile ou face $n$ fois.

1) 
Donner une repr\'esentation  de l'espace $\Omega$ des \'ev\'enements
\'el\'ementaires de cette exp\'erience. 

2) 
Ecrire l'\'{e}v\'{e}nement $F=$``pile n'a pas \'{e}t\'{e} obtenu lors 
des 2 premiers lancers'' comme sous-ensemble de $\Omega$.

3)
Soit $i\in \{1,\ldots,n\}$. 
D\'{e}crire les \'{e}l\'{e}ments de l'\'{e}v\'{e}nement $E_i=$ 
``le r\'{e}sultat du $i$-i\`{e}me lancer est pile''.

4) 
Ecrire \`{a} l'aide des \'{e}v\'{e}nements $E_i$ l'\'{e}v\'{e}nement $F$.

5) 
Ecrire \`{a} l'aide des \'{e}v\'{e}nements $E_i$ l'\'{e}v\'{e}nement 
$G=$`` la pi\`{e}ce est tomb\'{e}e au moins une fois sur pile''.

6) 
Ecrire \`{a} l'aide des ensembles $E_i$ l'\'{e}v\'{e}nement 
$H=$`` la pi\`{e}ce est tomb\'{e}e au moins deux fois sur pile''.\\



{\bf Exercice 4:} 


On tire, une \`a une, sans les remettre dans le paquet, cinq cartes dans un jeu de 32 cartes; 
chaque succession de cartes  ainsi tir\'ees s'appelle une main.

1) Quel est l'espace de probabilit� que vous consid�rez ?

2) Quelle est la probabilit� qu'une main  contienne :
\begin{itemize}
\item[a)] que des cartes d'une m\^eme couleur ?
\item[b)] exactement trois rois? au plus un roi? au moins un roi ou
un as?
\end{itemize}

3) M\^emes questions si le tirage se fait avec remise.\\


{\bf Exercice 5:}

 On rel\`eve les dates d'anniversaire dans un groupe de $30$
personnes (jours num\'erot\'es de $1$ \`a $365$). 
On suppose que les jours de naissance sont \'equiprobables. 
Quelle est la probabilit\'e que deux personnes au moins aient le
m\^eme anniversaire?\\


{\bf Exercice  6:}


1)
En d\'enombrant les fa\c{c}ons de tirer $n$ boules parmis $n$ boules blanches et $n$ boules
noires, montrer que:
$$ C_{2n}^n=\sum_{k=0}^n (C_n^k)^2.$$

2) Retrouver cette formule \`a partir de l'identit\'e $(1+t)^n(1+t)^n=(1+t)^{2n}$.

3) Deux personnes lancent chacune $n$ fois une pi\`ece de monnaie. Quelle
est la probabilit\'e
$p_n$ qu'elles obtiennent le m\^eme nombre de piles? 
{\em Indication:} On utilisera l'\'equiprobabilit\'e sur
l'espace $\Omega=\{{\tt f}, {\tt p}\}^{2n}$.

4) 
Donner un \'equivalent de $p_n$ quand $n$ tend vers $+\infty$.
{\em Indication:} On rappelle la formule de Stirling:
$$ n!=\sqrt{2\pi}n^{n+1/2}\exp{-n}(1+\epsilon_n),\quad\textrm{avec}\quad
\lim_{n\to +\infty}
\epsilon_n=0.$$

5) Que pensez vous de l'affirmation suivante: \emph{ Lorsqu'on jette un
grand nombre
(pair) de fois une pi\`ece \'equilibr\'ee, il y a une forte probabilit\'e d'obtenir
exactement
autant de piles que de faces} ?\\


{\bf Exercice  7:}


Une secr\'etaire un peu distraite a tap\'e $ N $ lettres et pr\'epar\'e
$ N $ enveloppes portant les adresses des destinataires, mais elle
r\'epartit au hasard les lettres dans les enveloppes. 
Pour mod\'eliser cette situation, on choisit comme espace probabilis\'e
$ \Omega_N $ ensemble de toutes les permutations sur $ \{ 1,\ldots,N \} $
muni de l'\'equiprobabilit\'e $ P_N $.
Pour $ 1 \le j \le N $, on note $ A_j $ l'\'ev\'enement 
{\em la $ j$-\`eme lettre se trouve dans la bonne enveloppe}.

1)
Calculer $ P_N(A_j) $.

2)
On fixe $ k $ entiers $ i_1 < i_2 < \cdots < i_k $ entre $ 1 $ et $ N $.
D\'enombrer toutes les permutations $ \sigma $ sur $ \{1,\ldots,N\} $
telles que $ \sigma(i_1)=i_1 $, $ \sigma(i_2)=i_2 $, \ldots, $ \sigma(i_k)=i_k $.
En d\'eduire $ P_N(  A_{i_1} \cap A_{i_2} \cap \cdots \cap A_{i_k} ) $.

3)
On note $ B $ l'\'ev\'enement 
{\em au moins une des lettres est dans la bonne enveloppe}.
Exprimer $ B $ \`a l'aide des $ A_j $.

4)
Utiliser la formule de Poincar\'e pour calculer $ P_N(B) $
et sa limite quand $ N $ tend vers l'infini.



\end{document}

