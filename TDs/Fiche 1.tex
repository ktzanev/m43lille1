\documentclass[a4paper,12pt,reqno]{amsart}
\usepackage{macros_M43}

\begin{document}

% ===================================================================
\hautdepage{Fiche 1: Espaces probabilisés}
% ===================================================================


%-----------------------------------
\begin{exo}

  Trois enfants lancent chacun un ballon en direction d'un panier de basket. Il est convenu que celui qui marquera gagnera un paquet de bonbons, et qu'en cas d'ex-æquo les vainqueurs se partageront le paquet. Si personne ne réussit son panier, chacun mangera le tiers des bonbons. En utilisant les événements:
  \begin{eqnarray*}
    A & = & \{\text{Arthur marque un panier}\},\\
    B & = & \{\text{Béatrice marque un panier}\},\\
    C & = & \{\text{Cécile marque un panier}\}.
  \end{eqnarray*}
  écrire de façon ensembliste les événements suivants:
  \begin{eqnarray*}
    D & = & \{\text{tous les trois réussissent à marquer}\},\\
    E & = & \{\text{aucun ne réussit à marquer}\},\\
    F & = & \{\text{Béatrice mange tous les bonbons}\},\\
    G & = & \{\text{les trois enfants mangent des bonbons}\},\\
    H & = & \{\text{Cécile mange au moins un bonbon}\},\\
    I & = & \{\text{Arthur ne reçoit aucun bonbon}\}.
  \end{eqnarray*}
  Parmi tous ces événements, lesquels sont des événements élémentaires?

\end{exo}


%-----------------------------------
\begin{exo}

  On  lance un dé jusqu'à la première apparition du six. Notons:
    $$
      S_i=\{\text{Le $i$-ième lancer donne un six}\},\qquad i\in\N.
    $$
  Écrire à l'aide des événements $S_i$ et $S_i^c$ l'événement
    $$
      A=\{\text{La première apparition du six a lieu \emph{après} le 5-ième lancer}\}.
    $$
  Est-ce le même événement que
    $$
      B=\{\text{Six n'apparaît pas au cours des 5 premiers lancers}\}?
    $$

\end{exo}


%-----------------------------------
\begin{exo}

    Soit $n\in \N^*$. On tire à pile ou face $n$ fois.

    \begin{enumerate}
      \item Donner une représentation  de l'espace $\Omega$ des événements
      élémentaires de cette expérience.

      \item Écrire l'événement $F=$\{pile n'a pas été obtenu lors
      des 2 premiers lancers\} comme sous-ensemble de $\Omega$.

      \item Soit $i\in \{1,\ldots,n\}$.
      Décrire les éléments de l'événement $E_i=$
      \{le résultat du $i$-ième lancer est pile\}.

      \item Écrire à l'aide des événements $E_i$ l'événement $F$.

      \item Écrire à l'aide des événements $E_i$ l'événement
      $G=$\{la pièce est tombée au moins une fois sur pile\}.

      \item Écrire à l'aide des ensembles $E_i$ l'événement
      $H=$\{la pièce est tombée au moins deux fois sur pile\}.
    \end{enumerate}

\end{exo}


%-----------------------------------
\begin{exo}

  On tire, une à une, sans les remettre dans le paquet, cinq cartes dans un jeu de 32 cartes; chaque succession de cartes  ainsi tirées s'appelle une main.
  \begin{enumerate}
    \item Quel est l'espace de probabilité que vous considérez ?
    \item Quelle est la probabilité qu'une main  contienne :
    \begin{enumerate}
      \item que des cartes d'une même couleur ?
      \item exactement trois rois? au plus un roi? au moins un roi ou un as?
    \end{enumerate}
    \item Mêmes questions si le tirage se fait avec remise.
  \end{enumerate}
\end{exo}


%-----------------------------------
\begin{exo}

  On relève les dates d'anniversaire dans un groupe de $30$ personnes (jours numérotés de $1$ à $365$). On suppose que les jours de naissance sont équiprobables. Quelle est la probabilité que deux personnes au moins aient le même anniversaire?

\end{exo}


%-----------------------------------
\begin{exo}

  \begin{enumerate}
    \item En dénombrant les façons de tirer $n$ boules parmi $n$ boules blanches et $n$ boules noires, montrer que:
      $$
        C_{2n}^n=\sum_{k=0}^n (C_n^k)^2.
      $$

    \item Retrouver cette formule à partir de l'identité $(1+t)^n(1+t)^n=(1+t)^{2n}$.
    \item Deux personnes lancent chacune $n$ fois une pièce de monnaie. Quelle est la probabilité $p_n$ qu'elles obtiennent le même nombre de piles?

    \begin{indication}
      On utilisera l'équiprobabilité sur l'espace $\Omega=\{{\tt f}, {\tt p}\}^{2n}$.
    \end{indication}

    \item Donner un équivalent de $p_n$ quand $n$ tend vers $+\infty$.

    \begin{indication}
      On rappelle la formule de Stirling:
        $$
          n!=\sqrt{2\pi n}\Big(\frac{n}{e}\Big)^{n}(1+\varepsilon_n),\quad\text{avec}\quad \lim_{n\to +\infty} \varepsilon_n=0.
        $$
    \end{indication}\hspace{-\baselineskip}

    \item Que pensez vous de l'affirmation suivante: \emph{Lorsqu'on jette un grand nombre (pair) de fois une pièce équilibrée, il y a une forte probabilité d'obtenir exactement autant de piles que de faces} ?
  \end{enumerate}

\end{exo}


%-----------------------------------
\begin{exo}

  Une secrétaire un peu distraite a tapé $ N $ lettres et préparé $ N $ enveloppes portant les adresses des destinataires, mais elle répartit au hasard les lettres dans les enveloppes. Pour modéliser cette situation, on choisit comme espace probabilisé $ \Omega_N $ ensemble de toutes les permutations sur $ \{ 1,\ldots,N \} $ muni de l'équiprobabilité $ P_N $. Pour $ 1 \le j \le N $, on note $ A_j $ l'événement \emph{la $ j$-ième lettre se trouve dans la bonne enveloppe}.
  \begin{enumerate}
    \item Calculer $ P_N(A_j) $.

    \item On fixe $ k $ entiers $ i_1 < i_2 < \cdots < i_k $ entre $ 1 $ et $ N $. Dénombrer toutes les permutations $ \sigma $ sur $ \{1,\ldots,N\} $ telles que $ \sigma(i_1)=i_1 $, $ \sigma(i_2)=i_2 $, \ldots, $ \sigma(i_k)=i_k $. En déduire $ P_N(  A_{i_1} \cap A_{i_2} \cap \cdots \cap A_{i_k} ) $.

    \item On note $ B $ l'événement \emph{au moins une des lettres est dans la bonne enveloppe}. Exprimer $ B $ à l'aide des $ A_j $.

    \item Utiliser la formule de Poincaré pour calculer $ P_N(B) $ et sa limite quand $ N $ tend vers l'infini.
  \end{enumerate}

\end{exo}

\end{document}

