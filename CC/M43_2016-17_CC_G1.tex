\documentclass[12pt,reqno]{amsart}
\usepackage{macros_M43}

\begin{document}

% ==================================
\hautdepage{%
Interrogation
\par\normalfont\normalsize
29 mars 2017\\{[ durée: 1 heure ]}\par
  }
% ==================================
\ifsolutions\else
\textit{Les documents et les calculatrices ne sont pas autorisés.}
\vspace{7mm}
\fi

\bigskip

%-----------------------------------
\begin{exo}

  Dans une poste d'un petit village, on remarque qu'entre 10 heures et 11 heures, la probabilité pour que deux personnes entrent durant la même minute est considérée comme nulle et que l'arrivée des personnes est indépendante de la minute considérée.\newline
  On a observé que la probabilité pour qu'une personne se présente entre la minute $n$ et la minute $n+1$ est $p = \dsfrac{1}{10}$.\newline
  On veut calculer la probabilité pour que $0,1,2,3,4,5,\dots$ personnes se présentent au guichet entre 10h et 11h.\newline
  Soit la variable aléatoire X = « nombre de personnes se présentant au guichet entre 10h et 11h ».
  \begin{enumerate}
    \item Déterminer la loi de  $X$.
    \item  Par quelle loi peut-on l'approcher ?
    \item En utilisant cette approximation, calculer les probabilités suivantes :\\
    $P(X=3)$, $P(X=4)$ et $P(X=5)$.

    \item Quelle est la probabilité pour qu'au moins 10 personnes se présentent au guichet entre 10h et 11h? \emph{(Donner la réponse sous forme d'une somme.)}
  \end{enumerate}
\end{exo}

\tsvp
\newpage

%-----------------------------------
\begin{exo}

 On note $ N$ la variable aléatoire comptant le nombre d'œufs qu'un insecte donné pond. On suppose que $N$ suit une loi de Poisson de paramètre $\lambda>0$. On suppose également que chaque œuf donne naissance à un nouvel insecte avec probabilité $p \in \,]0,1[$\,, indépendamment de l'éclosion des autres œufs et de la valeur de $N$. On note $D$ le nombre d'insectes éclos.
  \begin{enumerate}

    \item Sachant que l'insecte pond $5$ œufs, quelle est la probabilité que $3$ insectes éclosent ?

    \item Quelle est la loi de $D$ sachant que $\{N=n\}$ avec $n\ge 1$?

    \item En déduire que pour tout $(n, d)\in\mathbb{N}^2$
      $$
        P(D = d \text{ et } N = n)=
          \begin{dcases}
            0 & \text{ si } d>n \\
            \frac{(\lambda p)^d}{d!}
            ~e^{-\lambda}\,
            \frac{\big[\lambda (1-p)\big]^{\makebox[4pt][l]{$\scriptstyle n-d$}}}{(n-d)!}\quad
              & \text{ si } d\le n
          \end{dcases}\quad.
      $$

    \item Démontrer que $D$ suit une loi de Poisson de paramètre $p\lambda$.

    \item Soit $E$ la variable
      $$
        E=
          \begin{cases}
            \text{«peu»} & \text{si\quad} D < 4\\
            \text{«beaucoup»} & \text{sinon}
          \end{cases}\quad.
      $$
      Déterminer la loi de $E$.

  \end{enumerate}

\end{exo}
\end{document}


