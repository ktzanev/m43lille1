\documentclass[12pt,reqno]{amsart}
\usepackage{macros_M43}

\begin{document}

% ==================================
\hautdepage{%
Interrogation
\par\normalfont\normalsize
29 mars 2017\\{[ durée: 1 heure ]}\par
  }
% ==================================
\ifsolutions\else
\textit{Les documents et les calculatrices ne sont pas autorisés.}
\vspace{7mm}
\fi

\bigskip

%-----------------------------------
\begin{exo}
 Dans un lot de 100 composants électroniques, il y a deux composants d\'efectueux. On pr\'el\`eve au hasard sans remise $n$ composants dans ce lot et on note $X$ le nombre de composants d\'efectueux parmi les $n$ pr\'elev\'es.
\begin{enumerate}
\item On suppose que $2 \le n \le 98.$ Donner la loi de $X.$
 \item Exprimer le plus simplement possible  $P(X=2)$ lorsque $2 \le n \le 98.$
 \item Quelle est la loi de $X$ si $n=100$ ?
 \item Je choisis un composant au hasard. Quelle est la probabilit\'e qu'il soit d\'efectueux ?
 \item En d\'eduire la loi de $X$ si $n=1.$
 \item En d\'eduire aussi la loi de $X$ si $n=99.$
\end{enumerate}
 
 
\end{exo}

\tsvp
%-----------------------------------
\begin{exo}

On s'intéresse à la reproduction d'un insecte. On suppose que chacun de ses  œufs donne naissance à un nouvel insecte avec une probabilité $p \in \,]0,1[$\,, indépendamment du nombre d'œufs pondus et de l'éclosion des autres œufs.
\begin{enumerate}
 \item Si l'insecte a pondu $5$ œufs, quelle est la probabilité qu'exactement $3$ insectes éclosent ?
\end{enumerate}

 On note maintenant $ N$ la variable aléatoire comptant le nombre d'œufs qu'un insecte donné pond. On suppose que $N$ suit une loi de Poisson de paramètre $\lambda>0$.  On note $D$ le nombre d'insectes éclos.
  \begin{enumerate}\setcounter{enumi}{1}
  \item Quelle est la loi de $D$ sachant que $\{N=n\}$ avec $n\ge 1$?

    \item En déduire que pour tout $(n, d)\in\mathbb{N}^2$
      $$
        P(D = d \text{ et } N = n)=
          \begin{dcases}
            0 & \text{ si } d>n \\
            \frac{(\lambda p)^d}{d!}
            ~e^{-\lambda}\,
            \frac{\big[\lambda (1-p)\big]^{\makebox[4pt][l]{$\scriptstyle n-d$}}}{(n-d)!}\quad
              & \text{ si } d\le n
          \end{dcases}\quad.
      $$

    \item Démontrer que $D$ suit une loi de Poisson de paramètre $p\lambda$.

    \item Soit $E$ la variable
      $$
        E=
          \begin{cases}
            \text{«peu»} & \text{si\quad} D < 4\\
            \text{«beaucoup»} & \text{sinon}
          \end{cases}\quad.
      $$
      Déterminer la loi de $E$.

  \end{enumerate}

\end{exo}



\end{document}


